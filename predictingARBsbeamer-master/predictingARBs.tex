\documentclass[10pt,hyperref={CJKbookmarks=true},xcolor=dvipsnames,aspectratio=169]{beamer}
\usetheme[navigation]{UMONS}
\usepackage[utf8]{inputenc}
\usepackage{verbatim}
\usepackage{ctex}

\title[Predicting ARBs]{Predicting aging-related bugs using software complexity metrics}
\author[B.Jason]{Jason \textsc{Bury}}
\institute[]{%
 Faculté des Sciences\\
  Université de Mons
  \\[2ex]
  \includegraphics[height=4ex]{figures/UMONS}\hspace{2em}%
  \raisebox{-1ex}{\includegraphics[height=6ex]{figures/UMONS_FS}}
}

\begin{document}
\maketitle
\section{Introduction}
\subsection{Aging-related bugs}

\begin{frame}
 \frametitle{Aging-Related Bugs}
 \framesubtitle{Mandelbug}
 If
 \begin{itemize}
  \item A time lag between the fault activation and the failure occurence.
  \item Bug influenced by timing of operations or order of operations or environment interactions.
 \end{itemize}
 \vspace{1cm}
 \begin{tabular}{llll}
  Then & $\Longrightarrow$ & \emph{Mandelbug} & :-( \\
  Else & $\Longrightarrow$ & \emph{Bohrbugs} & :-) , can thus be easily isolated
 \end{tabular}
\end{frame}

\begin{frame}
 \frametitle{Aging-Related Bugs试试中文}
 \framesubtitle{Aging-related mandelbug}
 If Mandelbug and\\
 \vspace{1cm}
 The bug causes the accumulation of internal error states
 \begin{center}
  \large Or \normalsize
 \end{center}
 The running time increase the risk of the bug activation/propagation\\
 \vspace{1cm}
 Then $\Longrightarrow$ \emph{Aging-related Mandelbug} :'( \\
 \small Also called \emph{Aging-related bug} or \alert{ARB}.\normalsize
\end{frame}

\begin{frame}
 \frametitle{Aging-Related Bugs}
 \framesubtitle{Examples}
 \begin{exampleblock}{What can causes an ARB ?}
  \begin{itemize}
   \item Numerical error due to rounding or imprecision accrue over time
   \item A malloc() without free()
   \item A queue continuously increased
   \item Size of ressource exceeded
   \item ...
  \end{itemize}
 \end{exampleblock}
\end{frame}

\subsection{How to detect}
\begin{frame}
 \frametitle{How can we detect them ?}
 short answer: \alert{Select metrics then use a classifier !}\\
 But, there are remaining questions about that solution:\\
 \begin{block}{Questions}
  \begin{itemize}
   \item Which metrics to use ?
   \item New metrics ?
   \item Which classifier to use ?
   \item will be the classifier effective for all projects ?
  \end{itemize}
 \end{block}
\end{frame}

\begin{frame}
 \frametitle{Project used as training set}
 3 complex projects will be used.\\
 \begin{itemize}
  \item Linux
  \item MySQL
  \item CARDAMOM, a middleware for air traffic control systems
 \end{itemize}
 \vspace{0.2cm}
 ARBs are found manually.\\
\end{frame}
% middleware: un réseau d'échange d'informations entre différentes applications informatiques.

\input{classifiers}

\input{attributes}

\input{newMetrics}

\input{cross}

\section{Conclusion}
\subsection{Conclusion}
\begin{frame}
 \frametitle{Conclusion}
 Predicting ARBs with software metrics is possible.\\
 \vspace{1cm}
 High PD: We can avoid inspecting files classified as non ARB-prone.\\
 \vspace{1cm}
 We have to check if caracteristics are similar before using a classifier for another project.
\end{frame}

\begin{frame}{What's International Trade About?}




\begin{itemize}
	\item Do you have any expectation for this course?
	\item 迄今,你认为2018年经济领域发生的最significant的事是什么?
	\item 如果你想赢,你必须先知道它是如何运作的
\end{itemize}
\begin{block}{2017年中国贸易情况}
	2017年,中国\textbf{货物贸易总额27.79万亿人民币},折合\textbf{4.105万亿美元}。其中,出口15.33万亿元,增长10.8\%;进口12.46万亿美元,增长20.9\%;贸易顺差2.87万亿美元,收窄14.2\%;\\
	2017年,中国前三大贸易伙伴分别是\textbf{欧盟、美国和东盟},三者合计占我国进出口总值的41.8\%;\\
	2017年,中国\textbf{一般贸易}进出口15.66万亿元,占比56.4\%,贸易结构有所优化;\\
	2017年,中国出口最多的产品是\textbf{机电类产品},共出口8.95万亿,占比58.4\%,传统劳动密集型产品合计出口3.08万亿元,占比21.1\%;进口最多的三类商品是原油、铁矿石,汽车。\\
\end{block}
\end{frame}


\end{document}